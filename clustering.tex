\section{Parallelism for Regaining Performance}
\label{sec:clustering}

Deslinkski et al.~\cite{dreslinski2010near} claim ``In applications where there
is an abundance of thread-level parallelism the intention is to use 10 s to 100
s of NTC processor cores that will regain 10-–50X of the performance, while
remaining energy efficient.'' In order to regain the performance lost from using
near-threshold techniques, Zhai et al.~\cite{Zhai:2007kn} and Dreslinski et
al.~\cite{Dreslinski:2007id} present a technique for leveraging parallelism in
the NTC regime. The proposed architecture groups multiple cores into clusters
which share an L1 cache. This is motivated by the observation that SRAM has a
higher energy optimal $V_{dd}$ and $V_{th}$ than logic. Therefore, the energy
optimal frequency of an SRAM is higher than that of logic. Based on this
observation, the proposed technique shares the first-level cache with multiple,
slower cores.  The cache operates at $n$ times higher frequency than the cores,
where $n$ is the number of cores in a cluster.  Using this architecture, the
cores still maintain single-cycle memory accesses while the core and memory can
operate at their ideal $V_{dd}$ and $V_{th}$.

Using this technique, a 71\% energy savings over a baseline single core machine
on the highly parallel SPLASH2 benchmark is demonstrated. However, in
investigating these claims, some shortcomings with this approach are revealed.
Of primary concern is the large area overhead required to achieve the same
benchmark performance. In order to achieve the same performance as the single
core system, 6 cores and 3 times the baseline amount of cache were required. It
is important to remember that these results are being presented in comparison to
a single core reference on a highly parallel workload. By Amdahl's law, the use
of parallelism in this case will be most effective. It would require a larger
number of cores to achieve the same level of parallelism when comparing with a
baseline system that included more than a single core. In fact, to achieve the
same performance as the baseline, some benchmarks require as many as 16 cores
and 32 times as much L1 cache (\SI{2}{\mega\byte} vs \SI{64}{\kilo\byte}).

The clustering technique also uses separate $V_{th}$ tuning for the core and
cache to find the energy optimal voltage for the same performance. However, in
examining what these voltages actually are, it is revealed that neither $V_{dd}$
of the cache nor the logic is in the near-threshold regime, and both are in fact
operating at a $V_{dd}$ twice as high as the selected $V_{th}$. In modern
process technologies, the standard $V_{dd}$ already is approaching 2x $V_{th}$.

One final concern is that no co-optimized configuration is presented for all
benchmarks. This could potentially be an issue as the energy optimal range of
cores, clusters, cache sizes, $V_{dd}$ and $V_{th}$ is large. It is unclear what
the energy savings across benchmarks for different configurations will be.
