
\section{Device Optimizations}
Dreslinski et al. \cite{Dreslinski:2010ez} suggest modifications of the transistor structure to reduce delay by reducing inverse sub-threshold slope ($S_S$). This can take the form of either modifying the channel doping profile \cite{Paul:2004cx} or increasing oxide length \cite{Hanson:2007uu}.

Hanson shows that the main delay benefit from scaling $S_S$ comes from the assumption that the CMOS gate is operating at the minimum energy point, a point far into sub-threshold that is proportional to $S_S$. Taking this term away, his equation instead becomes

\begin{equation}
t_p \propto \frac{1}{e^\frac{V_{D}-V_{TH}}{S_S}}
\end{equation} 

which is a much weaker effect for $S_S$ in the range of 80-90mV. This is even worse because Hanson is using an equation that assumes we are far sub-threshold. As we pass$ V_{TH}$, the current and delay begin to instead scale with an interpolation between super-threshold and sub-threshold models rather than just the sub-threshold model. This means that the effect of $S_S$ becomes significantly weaker.

 Raychowdhury \cite{Raychowdhury:2006fu} shows that this sub threshold slope has less of an effect as $V_{DD}$ approaches $V_{TH}$.
 
 **Insert figure from slide 7 of presentation **
 