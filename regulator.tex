\section{Voltage Regulators}
\label{sec:regulators}

As shown in [A], compared to operating a digital design at the nominal voltage,
operating a design at a supply voltage level that is near the device threshold
voltage provides more energy-efficient computations. However, to fairly evaluate
the energy saving obtainable from the near-threshold computing approach, the
overhead in generating a near-threshold supply needs to be evaluated. Both [B]
and [C] raised the concern that the efficiency of the voltage regulator would be
lower for delivering power to a near-threshold computing design. [B] presents
evidence that typical voltage regulator efficiency is lower at light load and
argues that if a near-threshold design consumes less power the efficiency will
be lower.  [C] considers the case where the chip supply voltage is generated
with an on-chip switch-capacitor converter. They shows that the efficiency of
the regulator is at 90\% for an output voltage of \SI{1.2}{\volt} and is at 70\%
for an output voltage of \SI{0.5}{\volt}.

In the context of server computing, power is typically delivered to the
microprocessor through a buck converter which down converts the \SI{12}{\volt}
motherboard voltage [].Although it is difficult to derive a quantitative
comparison between the efficiency of a \SI{1.2}{\volt} output for an above
threshold design and a \SI{0.5}{\volt} output for a near-threshold design, it is
possible to consider the difference in requirement for these two regulators and
understand where the difference in efficiency may come in. It is intuitive to
compare the above threshold converter and the near-threshold converter at the
same percentage of voltage ripple requirement. However, as shown in equation X3,
the delay of a logic gate is more sensitive to the supply voltage in the
near-threshold region.  Thus, if voltage ripple requirement is the same, the
near-threshold DC-DC converter will introduce more delay variation in the
design. As performance variation has already been one of the key limiting
factors for near-threshold computing, it is desirable to maintain the same
variation level due to voltage ripple. Thus, it is necessary to have a stringent
voltage ripple requirement for a near-threshold voltage regulator. 

In addition, the current consumption of the two regulators will also be
different. One of the goals of employing near-threshold computing for servers is
to reduce power consumption while maintaining throughput. However, unless
near-threshold computing can achieve power reduction by a factor that is larger
than the supply voltage ratio, the current drawn by a near-threshold design is
going to be higher. This higher output current is particularly problematic for
the power loss due to the output rectification. The efficiency loss in the
output rectification is the ratio between the output voltage and the voltage
drop across the output transistor. Since the output voltage is small, the
regulator is more sensitive to voltage drop across the output transistor. On top
of that, the higher output current further increases the voltage drop across the
output stage.  

