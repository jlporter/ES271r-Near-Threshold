\section{Voltage Regulators}

We already know logic gates computes more efficiently at a supply voltage level
that is near the device threshold voltage. When we attempt to translate the
energy efficiency of these gates to the energy efficiency of a whole system, we
need to carefully evaluate the associated overhead.
Both~\cite{dreslinski2010near,Zhai:2007kn}
and~\cite{dreslinski2010near,Zhai:2007kn} raised the concern that the efficiency
of the voltage regulator would be lower for delivering power to a near-threshold
computing design. \cite{dreslinski2010near,Zhai:2007kn} presents evidence that
typical voltage regulator efficiency is lower at light load and argues that if a
near threshold design consumes less power the efficiency will be lower.
\cite{dreslinski2010near,Zhai:2007kn} considers the case where the chip supply
voltage is generated with an on-chip switch-capacitor converter. They shows that
the efficiency of the regulator is at 90\% for an output voltage of 1.2V and is
at 70\% for an output voltage of 0.5V.

In the context of server computing, the power is delivered to the
microprocessors typically through a buck converter which down converts the 12V
mother board voltage~\cite{dreslinski2010near,Zhai:2007kn}. Although it is
difficult to derive a quantitative comparison between the efficiency of a 1.2V
output for an above threshold design and a 0.5 output for a near threshold
design, it is possible to consider the difference in requirement for these two
regulators and understand where the difference in efficiency may come in. 

The performance of digital circuits is highly susceptible to power supply noise.
This susceptible is worse when the design is operating in the near threshold
region, as shown earlier in figure ~\ref{fig:voltage_delay}). On the other hand,
the power supply noise is a function of load stepping and decoupling capacitor.
Since load stepping is a function of the processor architecture, we can assume
simply scale down the supply voltage would not change the magnitude of the power
supply noise. This power supply noise is a larger percentage of the supply
voltage when the design is operating in near threshold region. As a result, it
cause more delay variation in the design. As performance variation has already
been one of the key limiting factors for near threshold computing, it become
necessary to suppress this voltage noise at the price of efficiency, board area
or chip area by increasing the feedback loop bandwidth and adding more
decoupling capacitors.

In addition,  one of the goals of employing near threshold computing for servers
is to reduce power consumption while maintaining throughput. However, unless
near threshold computing can achieve processor power reduction by a factor that
is larger than the factor supply voltage is reduced by, the current drawn by a
near threshold design is going to be higher. This higher output current is
particular problematic for the power loss due to the output rectification. The
efficiency loss in the output rectification is the ratio between the output
voltage and the voltage drop across the output transistor. Since the output
voltage is small, the regulator is more sensitive to voltage drop across the
output transistor. On top of that, the higher output current further increases
the voltage drop across the output stage.  
