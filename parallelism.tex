\section{Shortcomings of Application Parallelism}

Several recent works~\cite{Esmaeilzadeh2011Dark-silicon-an,Hardavellas:2011de}
have looked at the impact of dark silicon on computing in the near future. While
the number of transistors on a given die have been doubling every generation,
the number of transistors that can be powered for a fixed power budget have not
been increasing due to the slowing of transistor performance scaling. Since
power budgets have not been increasing past the limits defined by air cooling,
this has led to a situation where chips have more transistors than can be
powered at any given time. These unpowered transistors are referred to as dark
silicon.

Dark silicon seemingly presents an opportunity for near-threshold computing.
Deslinkski et al.~\cite{dreslinski2010near} state . ``More gates can now fit on
a die, but a growing fraction cannot actually be used due to strict power
limits.'' By reducing the power consumed per-core, more cores can be integrated
and simultaneously powered. Since these cores would be unpowered in a
super-threshold chip, this represents an opportunity for near-threshold chips to
regain performance compared to super-threshold operation.

While this dark silicon is an opportunity for devices to reach higher levels of
integration, recent studies analyzing dark silicon have not born out the promise
of higher levels of performance with increasing core counts.
Work by Esmaeilzadeh et. al~\cite{Esmaeilzadeh2011Dark-silicon-an}
developed an analytical model analyzing the impact of dark silicon for a range
of CMP configurations in future process technologies. They show projected
speedups for these configurations and process technologies using the PARSEC
benchmark suite, which represents workloads similar to the SPLASH2 benchmark
suite used in the previously discussed clustering architecture
papers~\cite{dreslinski2010near,Zhai:2007kn}, indicating that they are
targeting similar applications. This analytical model does project that
dark silicon will become a significant portion of CPUs in the near future,
dominating as soon as 2016 assuming conservative scaling parameters. However,
the authors also analyze the case where the power constraint is lifted, which
effectively removes the impact of dark silicon and allows more cores to be
powered.

