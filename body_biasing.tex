\section{Body Biasing}
\label{sec:bodybiasing}

Bodying biasing has been proven to be effective in compensating for the
performance loss of a design due to within-die systematic variation and
intra-die variation. However, it has been increasingly challenging to employ
this technique because the effect of body biasing is decreasing as transistor
technology scales to smaller dimension. In every technology generation, the gate
oxide capacitance $C_{ox}$ is scaled up for a factor 0.7 and the doping
concentration of the channel is scaled up by a factor of 0.7. By equation X1,
the combined effect of scaling these two parameters reduces the body effect
parameter gamma, which determines the effectiveness in using body biasing to
modify the device threshold voltage as shown in equation X2.

\begin{equation}
V_{T} = V_{TO} + \gamma ( \sqrt{ | {V_{SB} + 2\phi_{F} | } } - \sqrt{ | 2\phi_{F} | } 
\end{equation}

\begin{equation}
\gamma = (1/C_{ox})\sqrt{2q\epsilon_{si}N_A}
\end{equation}

On the other hand, systematic variation such as gate length is increasing with
every successive technology. [] shows that sigma of the critical dimension
variation caused by line edge roughness has been around 5nm and remained roughly
constant across different technology node. As a result, the variability in delay
continues to rise with technology scaling. 

Under these two trends, the need for body biasing compensation is increasing
while the effectiveness of body biasing is decreasing. Nevertheless, when the
body biasing technique is applied to a near threshold design, the delay tuning
range is much larger for the same amount of threshold modulation. Yet at the
same time, since a near-threshold design is more sensitive to process variation,
it is not obvious whether body biasing is more effective for a near threshold
design. 

To make a quantitative comparison between the effectiveness of body biasing for
both an above threshold design and a near threshold design across different
technology nodes,  we simulate the propagation  delay of a 6 FO4 inverter delay
chain in 32nm, 22nm and 16nm using ASU Technology Predictive Model []. To
simplify the analysis, the only variation considered is within-die systematic
and die-to-die gate length variation. Table X and table y shows the delay of the
inverter chain  without applying body biasing and with maximum forward body
biasing when no variation is presented. The results show that body biasing
indeed influence the delay of the inverter chain by a large percentage in the
near threshold region, but across technology node, the performance improvement
with maximum body biasing is decreasing.   

\begin{center}
  \begin{tabular}{ | l | l | l | l | }
    \hline
    & 32nm & 22nm & 16nm \\ \hline
    nominal delay(ps) & 45.0 & 26.0 & 17.2 \\ \hline
    maxiumum $V_{BBfwd}$ (ps)  & 35.5 & 21.1 & 14.2 \\  \hline
    percentage change  & 21\% & 18.9\% & 17.1\% \\ 
    \hline
  \end{tabular}
\end{center}


\begin{center}
  \begin{tabular}{ | l | l | l | l | }
    \hline
    & 32nm & 22nm & 16nm \\ \hline
    nominal delay(ps) & 1709 & 464 & 207 \\ \hline
    maxiumum $V_{BBfwd}$ (ps)  & 541 & 195 & 105 \\  \hline
    percentage change  & 69\% & 58\% & 49.3 \\ 
    \hline
  \end{tabular}
\end{center}

Table Z shows the result When we include variation in the model. The
effectiveness of process compensation by body biasing is measured in terms of
magnitude of gate length variation body biasing can restore. As suggested by
the model, although the compensation is very effective in 35nm for near
threshold computing, being able to compensate for 53\% of gate length variation,
this number will quickly be quickly be reduced to 11.8\% in two technology node.
At the same time, as argued earlier the gate length variation is increasing. In
future technology generation, the body biasing technique would not be sufficient
to offset the delay variation for near threshold computing.    

\begin{center}
  \begin{tabular}{ | l | l | l | l | }
    \hline
    & 32nm & 22nm & 16nm \\ \hline
    above Vt(nm) & 3.8 & 1.5 & 0.8 \\ \hline
    \% compnesation in L  & 10.8\% & 6.8\% & 5\% \\ \hline
    near Vt(nm)  & 17 & 4.3 & 1.9\\  \hline
    \% compnesation in L & 53\% & 19.5\% & 11.8\% \\ 
    \hline
  \end{tabular}
\end{center}
