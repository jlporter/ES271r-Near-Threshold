\section{Body Biasing}
\label{sec:bodybiasing}

Body biasing has been proven to be effective in compensating for the performance loss of a design due to process, temperature and voltage variations~\cite{BodyBiasing}.
However, it has been increasingly challenging to employ this technique because the effect of body biasing is decreasing as transistor technology scales to smaller dimension.
In every technology generation, both the gate oxide capacitance $C_{ox}$ and the doping concentration $N_A$ are scaled up by a factor of 0.7~\cite{Denard}.
By the equation,
\begin{equation}
\gamma = (1/C_{ox})\sqrt{2q\epsilon_{si}N_A}
\end{equation}
the combined effect of scaling these two parameters reduces the body effect parameter $\gamma$.
When $\gamma$ decreases, the influence of body biasing voltage $V_{SB}$ on device threshold voltage $V_{th}$ is subsequently reduced, as shown by
\begin{equation}
V_{T} = V_{TO} + \gamma ( \sqrt{ | {V_{SB} + 2\phi_{F} | } } - \sqrt{ | 2\phi_{F} | } )
\end{equation}
At the same time that the effectiveness of body biasing is decreasing, the need for techniques for dealing with device variation is increasing.
With various resolution enhancement techniques, the systematic variation in gate length has been kept to a steady percentage of the device gate length in recent technology nodes~\cite{Intel:2009}.
Intel shows that standard deviation of the system variation gate length variation is contained to approximately 3\% across different technology nodes~\cite{Intel:2009}.
However, due to the inherent resolution limitation in current lithography processes, maintaining the current levels of gate length variation in future technology generations will be increasingly challenging~\cite{OPC20}.
   
When body biasing is applied to a near-threshold design, the delay tuning range is much larger for the same amount of threshold modulation.
Yet at the same time, since a near-threshold design is more sensitive to process variation, it is not obvious whether body biasing is more effective for a near-threshold design.
To make a quantitative comparison between the effectiveness of body biasing for both a super-threshold design and a near-threshold design across different technology nodes, we simulate the propagation delay of a 6 FO4 inverter delay chain in \SI{32}{\nano\meter}, \SI{22}{\nano\meter} and \SI{16}{\nano\meter} using the ASU Predictive Technology Model~\cite{PredictiveModel}.
The HSPICE model also incorporates variations in gate length, threshold voltage, supply voltage, and temperature.
To prevent transistor latchup, the maximum forward body biasing is set to \SI{0.4}{\volt}~\cite{BodyBiasing}.
Table~\ref{tbl:body1} and Table~\ref{tbl:body2} show the delay of the inverter chain for zero body biasing and maximum forward body biasing when no variation is presented.
The results show that body biasing indeed influence the delay of the inverter chain by a large percentage in the near-threshold region, but across technology node, the performance improvement with maximum body biasing is decreasing.

\begin{table}
  \caption{Performance improvement by body-biasing in super-threshold computing} 
  \centering 
  \label{tbl:body1}
  \begin{tabular}{ | l | l | l | l | }
    \hline
    & \SI{32}{\nano\meter} & \SI{22}{\nano\meter} & \SI{16}{\nano\meter} \\ \hline
    Nominal delay (ps) & 45.0 & 26.0 & 30.4 \\ \hline
    Delay with body biasing (ps)  & 41.3 & 24 & 26.4 \\  \hline
    Percentage change  & 8.2\% & 7.7\% & 7.5\% \\ 
    \hline
  \end{tabular}
\end{table}



\begin{table}
  \caption {Performance improvement by body-biasing in near-threshold computing}  
  \centering
  \label{tbl:body2}
  \begin{tabular}{ | l | l | l | l | }
    \hline
    & \SI{32}{\nano\meter} & \SI{22}{\nano\meter} & \SI{16}{\nano\meter} \\ \hline
    Nominal delay (ps) & 1709 & 464 & 289 \\ \hline
    Delay with body biasing (ps)  & 889 & 278 & 149 \\  \hline
    Percentage change  & 48\% & 40\% & 33\% \\ 
    \hline
  \end{tabular}
\end{table}

On the other hand, the higher performance variation can also be observed by comparing the results in Table~\ref{tbl:body1} and Table~\ref{tbl:body2}. 

\begin{table}
  \caption{Performance variation in super-threshold computing} 
  \centering 
  \label{tbl:body3}
  \begin{tabular}{ | l | l | l | l | }
    \hline
    & \SI{32}{\nano\meter} & \SI{22}{\nano\meter} & \SI{16}{\nano\meter} \\ \hline
    Mean delay (ps) & 45.0 & 26.0 & 30.6 \\ \hline
    Standard Deviation (ps)  & 41.3 & 24 & 6.6 \\  \hline
    \hline
  \end{tabular}
\end{table}



\begin{table}
  \caption {Performance variation by body-biasing in near-threshold computing}  
  \centering
  \label{tbl:body4}
  \begin{tabular}{ | l | l | l | l | }
    \hline
    & \SI{32}{\nano\meter} & \SI{22}{\nano\meter} & \SI{16}{\nano\meter} \\ \hline
    Mean delay (ps) & 45.0 & 26.0 & 289 \\ \hline
    Standard Deviation (ps)  & 41.3 & 24 & 96 \\  \hline
    \hline
  \end{tabular}
\end{table}

To evaluate whether body biasing is sufficiently effective for compensating for the performance variation of a near-threshold design, the three standard deviation delay of the inverter with body biasing applied is calculated through simulation.
This delay is then used along with inverter chain nominal delay to compute the effective standard deviation after applying body biasing, denoted by $\sigma_{BB}$.
The result in Table~\ref{tbl:compensation} shows that body biasing is more effective in variation compensation in the near-threshold region.
However, even after the application of body biasing, the variation of a near-threshold design is still NX larger than a super-threshold  design in \SI{32}{\nano\meter}.
In future technology nodes, as the effectiveness of body biasing decreases, the factor of reduction in variation that can be achieved by using body biasing is also diminishing.
Therefore, body biasing alone is not sufficient to close the variation gap between super-threshold design and near-threshold design.
   

\begin{table}
  \caption {TBD}  
  \centering
  \label {tbl:compensation}
  \begin{tabular}{ | l | l | l | l | }
    \hline
    & \SI{32}{\nano\meter} & \SI{22}{\nano\meter} & \SI{16}{\nano\meter} \\ \hline
    $>V_{th}$ maximum variation allowed (nm) & 1.2 & 0.8 & 0.3 \\ \hline
    \% compensation in L  & 3.75\% & 3.6\% & 1.9\% \\ \hline
    $\sim V_{th}$ maximum variation allowed (nm)  & 6.8 & 1.9 & 0.9\\  \hline
    \% compensation in L & 21.25\% & 8.64\% & 5.62\% \\ 
    \hline
  \end{tabular}
\end{table}
