\section{Body Biasing}

Bodying biasing has been proven to be effective in compensating for the
performance loss of a design due to within-die systematic variation and
intra-die variation. However, it has been increasingly challenging to employ
this technique because the effect of body biasing is decreasing as transistor
technology scales to smaller dimension. In every technology generation, the gate
oxide capacitance Cox is scaled up for a factor 0.7 and the doping concentration
of the channel is scaled up by a factor of 0.7. By equation X1, the combined
effect of scaling these two parameters reduces the body effect parameter gamma,
which determines the effectiveness in using body biasing to modify the device
threshold voltage as shown in equation X2.

On the other hand, systematic variation such as gate length is increasing with
every successive technology. [] shows that sigma of the critical dimension
variation caused by line edge roughness has been around 5nm and remained roughly
constant across different technology node. As a result, the variability in delay
continues to rise with technology scaling. 

Under these two trends, the need for body biasing compensation is increasing
while the effectiveness of body biasing is decreasing. Nevertheless, when the
body biasing technique is applied to a near threshold design, the delay tuning
range is much larger for the same amount of threshold modulation. Yet at the
same time, since a near-threshold design is more sensitive to process variation,
it is not obvious whether body biasing is more effective for a near threshold
design. 

To make better comparison between the effectiveness of body biasing for both an
above threshold design and a near threshold design across different technology
nodes,  we simulate the propagation  delay of a 6 FO4 inverter delay chain in
32nm, 22nm and 16nm using ASU Technology Predictive Model []. To simplify the
analysis, the only variation considered is within-die systematic and die-to-die
gate length variation. The inverter chain is first simulated with no
body-biasing and no gate length variation to find the nominal delay. Forward
body biasing can restore the delay of the inverter chain to nominal value when
the inverter chain is slow as a result of gate length variation. While
maintaining the nominal delay, we can evaluate the amount of body biasng
necessary to compensate for a given magnitude gate length variation. The
effectiveness of body biasing at a given technology node is measured by the
percentage of gate length variation which bodying biasing is able to compensate
for. 

As shown in table[xx], although body biasing is much more effective for
near-threshold computing. It is still not sufficient to combat the every
increased gate length variation. Thus, body biasing cannot be the sole
technology to be relied on to solve the variability issues in near threshold
computing. 

