\section{Body Biasing}
\label{sec:bodybiasing}

Bodying biasing has been proven to be effective in compensating for the
performance loss of a design due to within-die systematic variation and
intra-die variation. However, it has been increasingly challenging to employ
this technique because the effect of body biasing is decreasing as transistor
technology scales to smaller dimension. In every technology generation, the gate
oxide capacitance $C_{ox}$ is scaled up for a factor 0.7 and the doping
concentration of the channel is scaled up by a factor of 0.7. By equation X1,
the combined effect of scaling these two parameters reduces the body effect
parameter gamma, which determines the effectiveness in using body biasing to
modify the device threshold voltage as shown in equation X2.

\begin{equation}
V_{T} = V_{TO} + \gamma ( \sqrt{ | {V_{SB} + 2\phi_{F} | } } - \sqrt{ | 2\phi_{F} | } 
\end{equation}

\begin{equation}
\gamma = (1/C_{ox})\sqrt{2q\epsilon_{si}N_A}
\end{equation}

With various resolution enhancement technique, the systematic variation in gate length has been kept to a steady percentage of the gate length variation in recent technology node~cit{Intel:2009}. Intel shows that sigma of the system variation gate length variation is managed to apprximately 3% across different technology node~cite{Intel:2009}. However, due to the inherent resolution limitation in current lithogothy process, percentage of gate length variation in future technology generation may start to increase again~cite{OPC20}.    


Under these two trends, the need for body biasing compensation is potentially increasing
while the effectiveness of body biasing is decreasing. Nevertheless, when the
body biasing technique is applied to a near threshold design, the delay tuning
range is much larger for the same amount of threshold modulation. Yet at the
same time, since a near-threshold design is more sensitive to process variation,
it is not obvious whether body biasing is more effective for a near threshold
design. 

To make a quantative comparison between the effectiveness of body biasing for both an
above threshold design and a near threshold design across different technology
nodes,  we simulate the propagation  delay of a 6 FO4 inverter delay chain in
32nm, 22nm and 16nm using ASU Technology Predictive Model \cite{PredictiveModel}. To simplify the
analysis, the only variation considered is within-die systematic variation. To limit the leakage power overhead to 100\% when body biasing is applied, the maxiumum forward body biasing is set to be 0.2V.  Table~\ref{body1} and table~\ref{body2} shows the delay of the inverter chain for zero body biasing and maxiumum forward body biasing when no variation is presented. The results show that body biasing indeed influence the delay of the inverter chain by a large percentage in the near threshold region, but across technolgoy node, the performance improvment with maxiumum body biasing is decreasing.   


\begin{table}
  \caption {Peformance improvement by body-biasing in above threshold computing} 
  \centering 
  \label {body1}
  \begin{tabular}{ | l | l | l | l | }
    \hline
    & 32nm & 22nm & 16nm \\ \hline
    nominal delay(ps) & 45.0 & 26.0 & 17.2 \\ \hline
    maxiumum $V_{BBfwd}$ (ps)  & 41.3 & 24 & 15.9 \\  \hline
    percentage change  & 8.2\% & 7.7\% & 7.5\% \\ 
    \hline
  \end{tabular}
\end{table}



\begin{table}
  \caption {Peformance improvement by body-biasing in near threshold computing}  
  \centering
  \label {body2}
  \begin{tabular}{ | l | l | l | l | }
    \hline
    & 32nm & 22nm & 16nm \\ \hline
    nominal delay(ps) & 1709 & 464 & 207 \\ \hline
    maxiumum $V_{BBfwd}$ (ps)  & 889 & 278 & 139 \\  \hline
    percentage change  & 48\% & 40\% & 33\% \\ 
    \hline
  \end{tabular}
\end{table}

Table~\ref{compensation} shows the result when variation is included in the model. The effectiveness of process compensation by body biasing is measured in terms of magnitude of gate length variation body biasing can restore. As suggestied by the model, although the compensation is very effective in 35nm for near threshold computing, being able to compensate for 21.25\% of gate length variation, this number will quickly be quickly be reduced to 5.6\% in two technolgoy node. As the sigma of systematic variation in gate length is approximately 3%, body biasing can only correct for 2 $\sigma$ in two generations. Thus, in future technology generation, the body biasing technique would not be sufficient to offset the delay variation for near threshold computing.    

\begin{table}
  \caption {Variation compensation by body-biasing}  
  \centering
  \label {compensation}
  \begin{tabular}{ | l | l | l | l | }
    \hline
    & 32nm & 22nm & 16nm \\ \hline
    above Vt(nm) & 1.2 & 0.8 & 0.3 \\ \hline
    \% compnesation in L  & 3.75\% & 3.6\% & 1.9\% \\ \hline
    near Vt(nm)  & 6.8 & 1.9 & 0.9\\  \hline
    \% compnesation in L & 21.25\% & 8.64\% & 5.62\% \\ 
    \hline
  \end{tabular}
\end{table}

