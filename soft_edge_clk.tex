 \section{Soft-edge clocking}
Another proposal has been the use of soft-edge clocking to increase speed in near-threshold devices by reducing the dependence on critical path delay \cite{Wieckowski:2008bo}.
 In this design, a traditional D-flip flop is modified to be driven by two offset clocks, generating a short transparency window.

The clock delay generation becomes an issue as variability increases.
 The simple method used in the paper of having a chain of inverters would be heavily susceptible to process variations.
 As the two clocks move closer to each other, the soft-edge effect becomes minimized; as the two clocks drift further apart, race-through becomes a concern.
 The frequency improvement of the SFF design over a standard DFF relies heavily on the size of this softness window, as shown in figure **INSERT FIGURE NUMBER**, but as $\sigma_v/\mu$ increases, the variance of the delay through this clock delay system also increases and the target transparency window of the flip flop gets driven more by the need to keep the SFF operating in the permitted region rather than the ideal softness window.

**INSERT FIGURE 13 FROM HANSON**
