\begin{abstract}

Near-threshold computing (NTC) has been suggested as a way to continue Moore's law scaling by offering a 10x improvement in energy efficiency~\cite{Dreslinski:2010ez}.
With this energy improvement however comes a range of problems including a 10x increase in delay (reduction in performance) and a 20x total performance uncertainty.
In this paper, we reexamine several techniques proposed for mitigating these issues, including device optimizations, soft-edge clocking and cluster-based architectures, and address why they may be less effective than stated at dealing with the issues of NTC.
We also examine the additional issue of DC-DC converter efficiency in NTC, and further show how the lack of application-level parallelism will provide limited opportunities for regaining performance.
We focus specifically on the context of general-purpose server computing, but many of the conclusions apply to other domains as well.

\end{abstract}
